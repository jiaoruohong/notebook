\textbf{删除排序数组中的重复项}\par

给定一个排序数组,你需要在原地删除重复出现的元素,使得每个元素只出现一次,返回移除后数组的新长度。\par

不要使用额外的数组空间,你必须在原地修改输入数组 并在使用 $ O(1) $ 额外空间的条件下完成。\par

说明:\par

为什么返回数值是整数,但输出的答案是数组呢?\par

请注意,输入数组是以引用方式传递的,这意味着在函数里修改输入数组对于调用者是可见的。\par

你可以想象内部操作如下:\par

\begin{lstlisting}[language=bash]
输入:nums = [-1,2,1,-4], target = 1

// nums 是以“引用”方式传递的。也就是说,不对实参做任何拷贝
int len = removeDuplicates(nums);

// 在函数里修改输入数组对于调用者是可见的。
// 根据你的函数返回的长度, 它会打印出数组中该长度范围内的所有元素。
for (int i = 0; i < len; i++) {
    print(nums[i]);
}
\end{lstlisting}
